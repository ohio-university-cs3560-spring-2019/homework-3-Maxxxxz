\documentclass{article}
\usepackage[utf8]{inputenc}

\title{CS3560 Homework 3: GDB}
\author{Michael Cooper}
\date{February 2019}

\begin

\maketitle


    The standard deviation function, $\sqrt{\frac{1}{N}\sum\limits_{i=1}^N(a_i - \mu)^2}$, determines the variation in a set of data, where $\mu$ is the mean, and $x_i$ is the data at the given index $i$, and $N$ is the number of data. To find it, you calculate the sum from 1 to $N$ of $(a_i - \mu)$, then square it. After that, you simply divide by the number of data, and take the square root of it.


\end
